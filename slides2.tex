\documentclass{beamer}
\usetheme{Madrid}
\usecolortheme{default}
\usepackage{listings}
\usepackage{xcolor}
\usepackage{graphicx}

% Define Jupyter-like syntax highlighting
\definecolor{jupyterblue}{RGB}{0,112,192}
\definecolor{jupytergreen}{RGB}{0,128,0}
\definecolor{jupyterorange}{RGB}{170,85,0}
\definecolor{jupyterpurple}{RGB}{128,0,128}
\definecolor{jupyterbackground}{RGB}{248,248,248}
\definecolor{jupyterborder}{RGB}{204,204,204}

\lstdefinestyle{jupyter}{
    backgroundcolor=\color{jupyterbackground},   
    commentstyle=\color{jupyterorange},
    keywordstyle=\color{jupyterblue},
    stringstyle=\color{jupytergreen},
    basicstyle=\ttfamily\small,
    breakatwhitespace=false,         
    breaklines=true,                 
    captionpos=b,                    
    keepspaces=true,                 
    numbers=none,                    
    showspaces=false,                
    showstringspaces=false,
    showtabs=false,                  
    tabsize=4,
    frame=single,
    framesep=5pt,
    frameround=tttt, % square corners
    framexleftmargin=5pt,
    framexrightmargin=5pt,
    framextopmargin=5pt,
    framexbottommargin=5pt,
    rulecolor=\color{jupyterborder}
}

\lstset{style=jupyter, language=Python}

% Adjust frame size and spacing
\setbeamertemplate{frametitle}[default][center]
\setbeamersize{text margin left=5mm,text margin right=5mm}

% Title slide information
\title{Python LLM DSLs}
\subtitle{Simplifying LLM Integration for SLAC Scientific Computing}
\author{Your Name}
\date{\today}

\begin{document}

% Title frame
\begin{frame}
\titlepage
\end{frame}

% Problem statement slide - split into two frames to prevent clipping
\begin{frame}[fragile]
\frametitle{The Problem: LLM Integration is Needlessly Complex}

\textbf{Standard Approach (LangChain):}
\begin{lstlisting}
from langchain_anthropic import ChatAnthropic
from langchain.prompts import PromptTemplate
from langchain.chains import LLMChain

llm = ChatAnthropic(anthropic_api_key=api_key, 
                    model="claude-3-opus-20240229")
template = """Answer with JSON {'answer': true/false}: 
              Is this EPICS PV in a valid state?"""
prompt = PromptTemplate(input_variables=[], template=template)
chain = LLMChain(llm=llm, prompt=prompt)
\end{lstlisting}
\end{frame}

\begin{frame}[fragile]
\frametitle{The Problem: LLM Integration is Needlessly Complex (cont.)}

\begin{lstlisting}
result = chain.run({})  # Parse JSON manually
if json.loads(result)["answer"]:
    print("EPICS PV state is valid")
\end{lstlisting}

\vspace{0.3cm}
\textbf{Why So Verbose?}
\begin{itemize}
\item \textbf{Wrong abstractions:} Chain-based design doesn't fit LCLS real-time needs
\item \textbf{Mixed concerns:} Prompt engineering, API calls, and parsing all exposed
\item \textbf{Mental overhead:} Too cumbersome for beamline scientists to use on the fly
\end{itemize}
\end{frame}

% Solution slide
\begin{frame}[fragile]
\frametitle{One Solution: Context-Sensitive Semantic DSL}

\begin{lstlisting}
# Our DSL approach for LCLS configuration checking
from lcls_llm import check_configuration

if check_configuration("Is PV:UNDULATOR:GAP in valid range?"):
    proceed_with_experiment()
else:
    alert_operator("Configuration issue detected")
\end{lstlisting}

\vspace{0.5cm}
\textbf{Key Innovation:} Semantic extension based on context
\begin{itemize}
\item Same function behaves differently in different contexts
\item LCLS-specific implementation details hidden from users
\item Code reads naturally for scientists and operators
\end{itemize}
\end{frame}

% How it works slide
\begin{frame}[fragile]
\frametitle{How It Works: Context-Sensitive Dispatch}

\begin{lstlisting}
# In a conditional context (fault detection):
if check_configuration("Is timing system properly configured?"):
    # Uses a JSON-based boolean template
    # Returns True/False for decision making
    start_beam_delivery()
\end{lstlisting}

\begin{lstlisting}
# In a variable assignment (knowledge assistant):
explanation = check_configuration("Explain timing system status")
# Uses a text-based template for detailed information
# Returns full explanation for operators
log_and_display(explanation)
\end{lstlisting}

\vspace{0.2cm}
The DSL detects the syntactic context and adapts behavior transparently.
\end{frame}

% Vision slide with SLAC-specific examples
\begin{frame}[fragile]
\frametitle{Vision: SLAC Facility-Specific LLM Primitives}

Implementing SLAC's roadmap with purpose-built DSLs:

\begin{lstlisting}
from slac_llm import check_epics, predict_anomaly, assist_timing

# Hypothetical example: S3DF anomaly detection
anomalies = predict_anomaly(beam_data)
if anomalies:
    recommend_actions(anomalies)

\end{lstlisting}
\end{frame}

% Benefits slide aligned with roadmap
\begin{frame}
\frametitle{Alignment with SLAC LLM Roadmap}

\textbf{Addresses Key Roadmap Categories:}
\begin{itemize}
\item \textbf{Knowledge Assistants:} DSLs for RAG-based information retrieval from eLog, Slack
\item \textbf{LCLS-specific Use Cases:} EPICS interpretation, timing system, configuration
\item \textbf{S3DF Use Cases:} Anomaly detection, fault prediction, troubleshooting
\item \textbf{Coding Assistants:} Analysis script generation for experimental data
\end{itemize}

\vspace{0.3cm}

\textbf{Implementation Benefits:}
\begin{itemize}
\item \textbf{Reduced learning curve:} Scientists use domain terminology, not LLM jargon
\item \textbf{Integration with SLAC systems:} Works with existing EPICS, data acquisition
\item \textbf{ROI enhancement:} Lower barriers to adoption accelerates value realization
\end{itemize}
\end{frame}

% Challenges slide with SLAC focus
\begin{frame}
\frametitle{Implementation Considerations for SLAC}

\begin{enumerate}
\item \textbf{Resource Requirements}
   \begin{itemize}
   \item DSLs need domain-specific training/fine-tuning
   \item Integration with secure SLAC systems requires careful planning
   \end{itemize}

\item \textbf{Development Priorities}
   \begin{itemize}
   \item Which roadmap items would benefit most from DSL approach?
   \item Begin with high-impact use cases (configuration, anomaly detection)
   \end{itemize}

\item \textbf{The Path Forward}
   \begin{itemize}
   \item Create working group with LCLS, S3DF representation
   \item Develop prototype for one key use case (e.g., EPICS assistant)
   \item Expand based on user feedback and roadmap priorities
   \end{itemize}
\end{enumerate}
\end{frame}

% Demo slide
\begin{frame}
\frametitle{Live Demo}

\begin{center}
\LARGE{Let's see how this works in practice...}

\vspace{1cm}
\textbf{[JUPYTER NOTEBOOK DEMO]}
\end{center}
\end{frame}

\end{document}
